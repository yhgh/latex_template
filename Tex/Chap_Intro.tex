\chapter{引言(绪论)}\label{chap:introduction}

\section{模版使用方法}

考虑到许多同学可能缺乏\LaTeX{}使用经验,本文首先介绍本模版的使用方法。论文推荐用overleaf在线编辑网站撰写,\url{https://www.overleaf.com/}。优点是可以通过Visual Editor模式进行所见即所得的撰写,并且可以一键编译,缺点是免费版编译时间有限制,如遇到编译超时问题,可以只保留正在编写的章节,注释掉其他章节,进行编译。全部撰写完后再进行尝试,或线下编译。论文主要撰写过程可分为下面几部分:
\begin{enumerate}
    \item 进入Tex文件夹中的Frontinfo.tex文件,填写相应信息。通过修改``degree”命令下的相关信息,如学位:学士、硕士、博士,论文模版会据学位名称自动切换,需确保拼写准确。
    \item 进入Tex文件夹中的Frontinfo.tex文件,填写摘要和关键词的中英文信息。
    \item 进入Tex文件夹中的Prematter.tex文件,填写符号和缩略语使用,如不需要进入根目录的Thesis.tex文件中,在句首用$\%$注释掉第81行“input$\{$Tex/Prematter$\}$”
    \item 为每个相关章节再Tex文件夹中创建空白的tex文件,如``Chap\_Intro.tex",和``Chap\_Guide.tex"等,再进入Tex文件夹中的Maimatter.tex文件中用``input"命令插入相应章节。
    \item 进入每个章节相应的tex文件中,撰写论文具体内容。
    \item 在Tex文件夹中的Backmatter.tex中填写发表成果和致谢信息。
\end{enumerate}

本毕业论文模版改编自莫晃锐的ucasthesis模版。
中国科学院大学学位论文模板ucasthesis基于中科院数学与系统科学研究院吴凌云研究员的CASthesis模板发展而来。如有需要,可通过下方链接下载ucasthesis模版
\begin{center}
    \href{https://github.com/mohuangrui/ucasthesis}{Github/ucasthesis}: \url{https://github.com/mohuangrui/ucasthesis}
\end{center}

\section{系统要求(如使用overleaf可忽略此节)}\label{sec:system}

\href{https://github.com/mohuangrui/ucasthesis}{\texttt{ucasthesis}} 宏包可以在目前主流的 \href{https://en.wikibooks.org/wiki/LaTeX/Introduction}{\LaTeX{}} 编译系统中使用,如\TeX{}Live和MiK\TeX{}。因C\TeX{}套装已停止维护,\textbf{不再建议使用} (请勿混淆C\TeX{}套装与ctex宏包。C\TeX{}套装是集成了许多\LaTeX{}组件的\LaTeX{}编译系统。 \href{https://ctan.org/pkg/ctex?lang=en}{ctex} 宏包如同ucasthesis,是\LaTeX{}命令集,其维护状态活跃,并被主流的\LaTeX{}编译系统默认集成,是几乎所有\LaTeX{}中文文档的核心架构)。推荐的 \href{https://en.wikibooks.org/wiki/LaTeX/Installation}{\LaTeX{}编译系统} 和 \href{https://en.wikibooks.org/wiki/LaTeX/Installation}{\LaTeX{}文本编辑器} 为
\begin{center}
    %\footnotesize% fontsize
    %\setlength{\tabcolsep}{4pt}% column separation
    %\renewcommand{\arraystretch}{1.5}% row space 
    \begin{tabular}{lcc}
        \hline
        %\multicolumn{num_of_cols_to_merge}{alignment}{contents} \\
        %\cline{i-j}% partial hline from column i to column j
        操作系统 & \LaTeX{}编译系统 & \LaTeX{}文本编辑器\\
        \hline
        Linux & \href{https://www.tug.org/texlive/acquire-netinstall.html}{\TeX{}Live Full} & \href{http://www.xm1math.net/texmaker/}{Texmaker} 或 Vim\\
        MacOS & \href{https://www.tug.org/mactex/}{Mac\TeX{} Full} & \href{http://www.xm1math.net/texmaker/}{Texmaker} 或 Texshop\\
        Windows & \href{https://www.tug.org/texlive/acquire-netinstall.html}{\TeX{}Live Full} 或 \href{https://miktex.org/download}{MiK\TeX{}} & \href{http://www.xm1math.net/texmaker/}{Texmaker}\\
        \hline
    \end{tabular}
\end{center}

\LaTeX{}编译系统,如\TeX{}Live(Mac\TeX{}为针对MacOS的\TeX{}Live),用于提供编译环境,\LaTeX{}文本编辑器 (如Texmaker) 用于编辑\TeX{}源文件。请从各软件官网下载安装程序,勿使用不明程序源。\textbf{\LaTeX{}编译系统和\LaTeX{}编辑器分别安装成功后,即完成了\LaTeX{}的系统配置},无需其他手动干预和配置。若系统原带有旧版的\LaTeX{}编译系统并想安装新版,请\textbf{先卸载干净旧版再安装新版}。

