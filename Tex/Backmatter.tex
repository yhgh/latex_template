%---------------------------------------------------------------------------%
%->> Backmatter
%---------------------------------------------------------------------------%
\chapter[后记]{后\qquad 记}\chaptermark{后\qquad 记}% syntax: \chapter[目录]{标题}\chaptermark{页眉}
%\thispagestyle{noheaderstyle}% 如果需要移除当前页的页眉
%\pagestyle{noheaderstyle}% 如果需要移除整章的页眉

致谢内容



\chapter{在学期间取得创新性成果情况}
\bigskip
\begin{table}[!htb]
  \centering
\resizebox{\textwidth}{!}
{\begin{tabular}{| c| c | c | c | c |}
\hline
       \textbf{成果名称 }&  \textbf{成果类别} &   \textbf{刊物名称/出版物名称} &  \textbf{刊发时间} & \textbf{作者次序}\\
\hline %\multicolumn{1}{|m{4cm}|}
      \makecell[c]{English article name \\of the published paper} &学术论文 & 《期刊名或缩写》 & 20xx\text{年第}x\text{卷(期)} & 1 \\%\text{年第}3\text{期}
\hline
      \makecell[c]{English article name \\of the published paper} & 学术论文 &《期刊名或缩写.》 & 20xx年第x卷(期) & 1 \\
\hline

\end{tabular}}
\end{table}

% \cleardoublepage%在“创新性成果页”后添加带页眉页脚的空白页(主要为方便打印纸质版),如需要请取消注释


% \section*{申请或已获得的专利:}

% (无专利时此项不必列出)

% \section*{参加的研究项目及获奖情况:}

% 可以随意添加新的条目或是结构。


% \chapter{导师(组)对学位论文的评语}


% \chapter{答辩委员会决议书}


% \chapter{作者简历:}

% \subsection*{casthesis作者}

% 吴凌云,福建省屏南县人,中国科学院数学与系统科学研究院博士研究生。

% \subsection*{ucasthesis作者}

% 莫晃锐,湖南省湘潭县人,中国科学院力学研究所硕士研究生。

% \cleardoublepage[plain]% 让文档总是结束于偶数页,可根据需要设定页眉页脚样式,如 [noheaderstyle]
%---------------------------以下为封底内容---------------------------------------------%
\cleardoublepage[empty]
    \thispagestyle{empty}
    \begin{center}
    \vspace*{350pt}
        {\yhei\zihao{3}勤\protect\hspace{1em}奋\protect\hspace{1em}创\protect\hspace{1em}新\protect\hspace{2em}为\protect\hspace{1em}人\protect\hspace{1em}师\protect\hspace{1em}表}\\  
        \end{center}
